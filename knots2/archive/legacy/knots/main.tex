
%\documentclass{tran-l}
%\documentclass[twocolumn]{amsart}
%\documentclass[]{amsart}
\documentclass[]{llncs}
\usepackage[all,knot]{xy}



\newif\ifpdf
\ifx\pdfoutput\undefined
\pdffalse % we are not running PDFLaTeX
\else
\pdfoutput=1 % we are running PDFLaTeX
\pdftrue
\fi

\ifpdf
\usepackage[pdftex]{graphicx}
\else
\usepackage{graphicx}
\fi
%\usepackage[pdftex]{graphicx}

%\documentclass[]{entcs}
%\usepackage[]{prentcsmacro}

%\usepackage[active]{srcltx} % SRC Specials for DVI Searching
\usepackage {mathpartir}
%\usepackage {listings}
%\usepackage {array}
\usepackage{url}

% From Allen's stable.
%\usepackage{bigpage}
\usepackage{bcprules}
\usepackage{code}
\usepackage{amsfonts}
\usepackage{amstext}
\usepackage{latexsym}
\usepackage{amssymb}
%\usepackage{caption}
\usepackage{multicol}

\input{lgmHeader.tex}

\newcommand{\papertitle}{Knots as processes: a new kind of invariant}
% use static date to preserve date of actual publication
\newcommand{\paperversion}{Draft Version 0.1 - December 19, 2004}

\newenvironment{toc}
{
\begin{list}{}{
   \setlength{\leftmargin}{0.4in}
   \setlength{\rightmargin}{0.6in}
   \setlength{\parskip}{0pt}
 } \item }
{\end{list}}

\newenvironment{narrow}
{
\begin{list}{}{
   \setlength{\leftmargin}{0.4in}
   \setlength{\rightmargin}{0.6in}
 } \item }
{\end{list}}

\def\lastname{Meredith and Snyder}

%The following makes sure the intro's quote of%%
%first theorem doesn't screw up the numbering

\newtheorem{thm}{Theorem} 
%%% ----------------------------------------------------------------------

\begin{document}

%\begin{frontmatter}
\title{Knots as processes: knot theory embeds in the $\pi$-calculus}
\titlerunning{Namespace logic}

\author{ L.G. Meredith\inst{1} \and David F. Snyder\inst{2} }
\institute{ Biosimilarity\\ 505 N72nd St, Seattle, WA 98103, USA, \\
  \email{ lgreg.meredith@gmail.com } \\
  \and Department of Mathematics\\ Texas State
University--San Marcos\\ 601 University Drive \\
San Marcos, TX 78666 \\
  \email{ dsnyder@txstate.edu }
} 

\maketitle              % typeset the title of the contribution

%%% ----------------------------------------------------------------------

\begin{abstract}

          We exhibit an encoding of knots into processes in the
          $\pi$-calculus such that knots are ambient isotopic if and
          only their encodings are weakly bisimilar.

\end{abstract}

% \begin{keyword}
% concurrency, message-passing, process calculus, reflection, program logic
% \end{keyword}

%\end{frontmatter}

\section{Introduction and motivation}

Recent research in concurrency theory has led many to consider
geometric interpretations of concurrent computation such as Goubault's
investigations of higher-order automata
\cite{DBLP:conf/concur/GoubaultJ92}
\cite{DBLP:journals/mscs/Goubault00a} and Herlihy's application of
homology theory \cite{DBLP:journals/entcs/HerlihyRT02}. Exploiting a duality hidden within that framework,
the authors have switched the roles of the domains
and the
tool of investigation,  returning to a
more traditional mathematical activity of
finding invariant features of spatial entities. 
More specifically, in this paper we use the\pic
 to construct an isotopy invariant of knots,
exhibiting a  \pic encoding of any
given knot $K$
as a process $\meaningof{K}_{\pi}$. The encoding is dependent upon the
knot presentation. Nonetheless, the encoding has
the property that two knots are isotopic
if and only if their encodings as processes are weakly
bisimilar.  Our main theorem is: 
\begin{thm}[Main Theorem]
  Two knots, $K_0$ and $K_1$ are ambient isotopic, written here $K_0
  \sim K_1$, iff their encodings as processes are weakly bisimilar,
  i.e.
  
  \[ % Don't number this equation
    K_0 \sim K_1 \iff \meaningof{K_0}_{\pi}
\simeq \meaningof{K_1}_{\pi}
  \]
\end{thm}

While this makes clear that the
encoding is an invariant up to isotopy,
 we see another motivation for investigating knots from
this perspective. Bisimulation has proven to be an
powerful and flexible proof principle, adaptable to a wide
range of situations and admitting a few
significant,
potent equivalence class ("up-to") techniques
\cite{DBLP:conf/lics/Sangiorgi04}
\cite{DBLP:conf/mfcs/Sangiorgi95}. 
Thus, we
seek to apply  to
topological inquiries what we have learned
from the past
several decades of investigation into notions of
the equivalence of computational behavior.
The dual notion, laid bare here in its
simplest-to-grasp context,
may bear
fruit in the study of the equivalence of spaces
which have simplicial models.

\subsection{Paper summary}

In section \ref{Pinut} we give a brief review of the polyadic $\pi$-calculus,
immediately following that with a review of the primary results from
knot theory needed to state and prove our main theorem. Next, we
introduce the intuitions behind the encoding, sketching the general
shape and walking through the procedure in the case of the trefoil. We
follow this with a section on the details of the
encoding. This puts us in a
position to state and prove the main theorem in section
\ref{MainThm}. In the penultimate section we discussion some of the results
following from this method of encoding, illustrating a way to
interpret knot composition as parallel composition and a way to
interpret the Kauffman bracket (and hence a number of other knot
invariants) in this setting. In the final section
we state some conclusions and
foreshadow some directions for future research.

\section{The $\pi$-calculus in a nutshell} \label{Pinut}

While the rapidly expanding literature on
mobile process calculi is both wide and
deep, unfortunately it does not as of yet
offer a
gentle introduction for the investigator who is not already steeped in
the mathematical culture of programming language semantics. Due
to space constraints we cannot rectify the situation here. We will,
however attempt to provide guiding intuition where possible.

\subsection{Note for
mathematicians}\label{mathnote}

What follows is to aid mathematicians in reading
the next two sections. Since
Milner's seminal paper
\textit{Functions as processes}
\cite{FunctionsAsProcesses} \footnote{which
title inspires this paper's.},
 process calculi are typically presented in a
manner mimicking the
presentation of an algebra via operators, generators
and relations. The grammar
generating the set of processes may be seen as a generalization of
generators and the structural equivalence may be seen as the relations
over the freely generated set, using the
allowable operation(s).
\footnote{This view is somewhat
  non-standard from the process calculi literature where structural
  equivalence is really a stepping stone and the equational theories
  considered are built to coincide with bisimulation.}

In this sense, the process calculi fit loosely into
the standard zoo of
algebraic structures, resembling vector spaces in the sense that they
are built out of two kinds of fundamental building blocks --
i.e. vector spaces are built out of scalars and
vectors (and the operations amongst these) while
process calculi are built out of names and processes.
What principally
distinguishes these calculi from classical
algebraic structures is
the introduction of an additional relation, called reduction, that is
neither a priori reflexive or symmetric, representing the evolution or
dynamics of a process. Thus, these structures have an explicit
representation of computation, unlike structures such as vector spaces
where dynamics is introduced via functionals
between such structures.

It is noteworthy that several other `computational calculi' also admit
similar algebraic presentations. The lambda calculi, as a family,
provide a canonical example: term grammar as generalized generators;
alpha-equivalence as relations; and beta-reduction as the explicit
account of dynamics. We submit that, by utilizing
bisimulation techniques, some such
computational calculi may generally constitute an
interesting mine for
topological invariants.

\subsection{\pic}

As mentioned above, the grammar generating the set of processes may be
seen as a kind of generalized set of generators in a presentation of
an algebra via generators and relations.

\begin{grammar}
\mbox{\ \textbullet\ summation} & {N} & \bc &
\Sigma_{i
\in I} x_i.A_i \\
\mbox{\ \textbullet\ agent} & {A} & \bc & F \;| \;
C \;| \; (\nu \; \vec{x})A \\
\mbox{\ \textbullet\ abstraction} & {F} & \bc &
(\vec{x})P \;| \; (\nu \; \vec{x})F \\
\mbox{\ \textbullet\ concretion} & {C} & \bc &
[\vec{x}]P \;| \; (\nu \; \vec{x})C \\
\mbox{\ \textbullet\ process} & {P,Q} & \bc & N
\;| \;P|Q \;| X\langle \vec{y} \rangle \;| \; (\textsf{rec} \; X(\vec{x}).P)\langle \vec{y} \rangle \;| \; (\nu \; \vec{x})P
\end{grammar} 

\textsc{ what is rec?}

Note, we identify summation over an empty index set with the null
process, $\pzero$, and prefixing with a summation over a singleton
index set. We adopt vector notation, $\vec{x}$, for finite sequences
of names, $x_0,...,x_{n-1}$, and define the \emph{arity} of sequences,
$|x_0,...,x_{n-1}| = n$, extending it in a natural way to abstractions
and concretions, e.g. $|(\vec{x})P| = |\vec{x}|$.

Further, we define \emph{pseudo-application} of an abstraction to a concretion via

\begin{equation}
  (\vec{y})P \circ (\nu \vec{v})[\vec{z}]Q \triangleq (\nu \vec{v})(P\{\vec{z}/\vec{y}\} | Q)
\end{equation}

provided that $\vec{y} \cap \vec{v} = \emptyset$ and $|\vec{y}| = |\vec{z}|$.

We also adopt the following standard abbreviations.

\begin{eqnarray}
  x?(\vec{y}).P & \triangleq & x.(\vec{y})P \\
  x!(\vec{y}).P & \triangleq & x.[\vec{y}]P
\end{eqnarray}

We also define: 

\begin{equation}
  X(\vec{y}) := P \triangleq (\vec{y})(\textsf{rec} \; X(\vec{x}).P)\langle \vec{y} \rangle
\end{equation}

\subsection{Structural congruence}

In keeping with the generalized generators and relations
view espoused in \ref{mathnote}, the
structural congruence can be seen as the set of relations over the
`free' algebra generated by the grammar above.

\begin{definition}
  The {\em structural congruence}, $\equiv$, between processes is the
  least congruence closed with respect to alpha-renaming, satisfying
  the abelian monoid laws (associativity, commutativity and $\pzero$
  as identity) for parallel composition as well as summation, and the
  following axioms:
\begin{enumerate}
\item the scope laws:
\begin{eqnarray}
 (\nu \; x)\pzero  & \equiv & \pzero, \nonumber\\
 (\nu \; x)(\nu \; x)P & \equiv & (\nu \; x)P, \nonumber\\
 (\nu \; x)(\nu \; y)P & \equiv & (\nu \; y)(\nu \; x)P, \nonumber\\
 P | (\nu \; x)Q & \equiv & (\nu \; x)(P|Q), \; \mbox{\textit{if} }x \not\in \freenames{P} \nonumber
\end{eqnarray}
\item
the recursion law:
\begin{eqnarray}
  (\textsf{rec} \; X(\vec{x}).P)\langle \vec{y} \rangle \equiv P\{\vec{y}/\vec{x}\}\{(\textsf{rec} \; X(\vec{x}).P)/X\} \nonumber
\end{eqnarray}
\end{enumerate}
\end{definition}

Note that the third scoping law inspires the
shorthand $(\nu\; xy)\triangleq  (\nu \; x)(\nu \;
y)$.
\subsection{Operational semantics} 

Finally, we introduce of the computational dynamics through the
reduction relation $\red$.

\infrule[Comm]
{ |F| = |C| }
{ x.F \juxtap x.C \red F \circ C }

In addition, we have the following context rules:

\infrule[Par]{{P} \red {P}'}{{{P} | {Q}} \red {{P}' | {Q}}}

\infrule[New]{{P} \red {P}'}{{\newp{{x}}{{P}}} \red {\newp{{x}}{{P}'}}}

\infrule[Equiv]{{{P} \scong {P}'} \andalso {{P}' \red {Q}'} \andalso {{Q}' \scong {Q}}}{{P} \red {Q}}

We write $\wred$ for $\red^*$.

\subsection{Bisimulation}

The computational dynamics gives rise to another kind of equivalence,
the equivalence of computational behavior. As previously mentioned
this is typically captured via some form of bisimulation. The notion
we use in this paper is weak barbed bisimulation \cite{milner91polyadicpi}.

% \begin{definition}
%   An \emph{observation relation}, $\downarrow$ is the smallest
%   relation satisfying the rules below.

% \infrule[Out-barb]{}
% {\outputp{x}{v} \downarrow x}
% \infrule[Par-barb]{\mbox{$P\downarrow x$ or $Q\downarrow x$}}
% {P|Q \downarrow x}

% We write $P \Downarrow x$ if there is $Q$ such that 
% $P \wred Q$ and $Q \downarrow x$.
% \end{definition}

\begin{definition}
  An agent, $B$, occurs \emph{unguarded} in $A$ if it has an occurence
  in $A$ not guarded by a prefix $x$. A process $P$ is observable at
  $x$, written here $P \downarrow x$, if some agent $x.A$ occurs
  unguarded in $P$. We write $P \Downarrow x$ if there is $Q$ such
  that $P \wred Q$ and $Q \downarrow x$.
\end{definition}

\begin{definition}
%\label{def.bbisim}
A \emph{barbed bisimulation} is a symmetric binary relation 
${\mathcal S}$ between agents such that $P\rel{S}Q$ implies:
\begin{enumerate}
\item If $P \red P'$ then $Q \wred Q'$ and $P'\rel{S} Q'$.
\item If $P\downarrow x$, then $Q\Downarrow x$.
\end{enumerate}
$P$ is barbed bisimilar to $Q$, written
$P \simeq Q$, if $P \rel{S} Q$ for some barbed bisimulation ${\mathcal S}$.
\end{definition}

One of the principal advantages of this framework is the co-algebraic
proof method for establishing bisimilarity between two processes:
exhibit a bisimulation \cite{DBLP:conf/lics/Sangiorgi04}.

\subsection{Linking abstractions and other process constructions} \label{linkingabs}

In the sequel we will find it useful to adapt some of Milner's
additional process constructions, such as the one for linking
abstractions. Given abstractions $F = (\vec{x})P$, and $G =
(\vec{y})Q$ with $|F|, |G| \geq j$ we may compose them, gluing
channels $x_{|F|-{j+1}},...,x_{|F|-1}$ to channels
$y_{0},...,y_{i'+j}$ by

\begin{eqnarray}
  & F \#_j G & \nonumber \\
  & \triangleq & \nonumber \\
  & (z_0,...,z_{|F|+|G|-(j+1)})(F\langle z_0,...,z_{|F|-1} \rangle | G\langle z_{|F|-{j+1}},..., z_{|F|+|G|-(j+1)}\rangle) &
\end{eqnarray}

We may omit the subscript, $_j$, when it is clear from context. Note
that the operation is associative, allowing us to write $F\#G\#H$
unambiguously.

Additionally, for an abstraction of the form, $F =
(\vec{z})\Pi_{i=0}^kP_i\langle z_{f(i,0)},...,z_{f(i,n_i)} \rangle$ with
$f$ and indexing function into the formals, $\vec{z}$, of the
abstraction, we write

\begin{eqnarray}
  F - P_j \triangleq (\vec{v})\Pi_{i=0}^{j-1}P_i\langle v_{f(i,0)},...,v_{f(i,n_i)}\rangle|(\vec{v})\Pi_{i=j+1}^{k}P_i\langle v_{f(i,0)},...,v_{f(i,n_i)} \rangle
\end{eqnarray}

to denote the deletion of the process $P_j$ from the parallel
composition. Thus, we have

\begin{eqnarray} \label{linkingident}
  F \equiv (F - P_j)\#P_j
\end{eqnarray}

\section{Knots}

In this section, we review a bit of knot theory and
establish some connections useful for proving our lemmas 
and theorems. By a
\textit{knot}, we mean an embedding, up to isotopy
equivalence, of the unit circle into three-dimensional
Euclidean space. Whether two given knots are of the same
isotopy class or not, and how this can be detected, is the
primary interest in knot theory. The most common way of
representing a knot is as a carefully chosen 
orthogonal projection of the knot onto a plane, where any
crossing point in the resulting diagram corresponds to at
most two points of the knot
\cite{LivingstonText}.  Reidemeister had the insight that
isotopies of maps can be decomposed into a sequence of
successive isotopies which, when projected,  result in
simple moves being performed upon a small region of the
knot diagram.

\subsection{Reidemeister moves}

There are three basic Reidemeister moves that may be
performed upon a knot
projection, which we denote by $\Omega_1, \Omega_2, \Omega_3$. These
are illustrated in the Fig. \ref{RMoves}. Note that each
move has an inverse move to ``undo" it, and there is a
mirror-image of each move.

\begin{figure}
\centering
\includegraphics[width=3in]{Reidemeister123.pdf}
\caption{The three Reidemeister moves}
\label{RMoves}
\end{figure}

Reidemeister established that two knot presentations are
ambient isotopic if
and only if there is a sequence of Reidemeister moves transforming one
presentation to the other (\cite{sossinskyknotsandbraids}). This
allows us to establish the following useful little lemma.

We first establish a lemma to our purposes, that shows we can clean up our knot diagram in a particularly suitable way.

Given a knot $K$ and a diagram $D(K)$ of $K$, we  call the application of a 
Reidemeister move $\rho$ a {\em neatening} move if the resulting diagram $D'(K)$ has a crossing number less than or
equal to the crossing number of $K$ i.e. any neatening move
of the type $\Omega_{1}$ or $\Omega_{2}$ can occur in
only one direction. Let $\hat{\rho}=\rho_{1}\cdots\rho_{n}$ be a sequence of successive Reidemeister moves on the
diagram $D(K)$ (with $n=0$ being the identity move). We call $\hat{\rho}$ a {\em neatening isotopy} if either
$n=0$ or  each
$\rho_{i}$
is a neatening move. If, in addition, at least one of the $\rho_{i}$ is a move of  type $\Omega_{1}$ or $\Omega_{2}$, we
call $\hat{\rho}$ a {\em cleaning isotopy} of $D(K)$. If $\rho_{n}$ is  move of  type $\Omega_{1}$ or $\Omega_{2}$, we
call $\hat{\rho}$ a {\em concise}  cleaning isotopy of $D(K)$

For a given diagram $D(K)$, the collection of all its neatening isotopies can be given a partial ordering: given
neatening isotopies $\hat{\rho}=\rho_{1}\cdots\rho_{n}$ and $\hat{\sigma}=\sigma_{1}\cdots\sigma_{m}$, we put
$\hat{\rho}\le\hat{\sigma}$ when $n\le m$ and $\rho_{i}=\sigma_{i}$  for all $i\le n$. This partial order restricts to the
the subcollection  of concise cleaning isotopies.

\begin{lemma}\label{simplificationLemma} For any knot  $K$
and knot diagram $D(K)$, we may reduce
$D(K)$ via a neatening isotopy to a diagram
$D'(K)$  having the property that no neatening isotopy of $D'(K)$ is a cleaning isotopy of $D'(K)$ .
\end{lemma}
\begin{proof} If the collection of cleaning isotopies of $D(K)$ is an empty one, then $D(K)$ is the desired diagram.
So assume there is at least one cleaning isotopy of $D(K)$.

The collection of concise cleaning isotopies of
$D(K)$
forms a partially ordered set. Since each
concise cleaning
isotopy reduces the
number of crossings in the diagram by at least 1, each chain in the partially order set has length bounded by the
number of crossings in $D(K)$. Select any maximal element $\hat{\rho}$ of the partially ordered set. Let $D'(K)$ be
the diagram derived from $D(K)$ by applying $\hat{\rho}$. 
\end{proof}



The lemma allows us to assume without loss of generality
that
we are working with minimal crossing diagrams.

\subsection{The Dowker-Thistlethwaite code}

The Dowker-Thistlethwaite code (``DT'') of a knot $K$  is obtained as follows:


Choose a point on $K$ and begin traversing  $K$ , counting each 
 crossing you pass through. If $K$
has $n$  crossings, then (since  every crossing is visited twice) the
count ends at $2n$. Label each crossing with the value of the counter
when it is visited (each crossing is labeled twice).
 Finally, when labeling a crossing with
 an even number,  prepend with the label
with a minus sign if traversing ``under" the crossing.
All crossings end up being labeled by a pair of integers whose absolute values
run, \emph{in toto}, from $1$ to $2n$. It is easy to see that each
 crossing is labeled with 
one odd integer and  one even integer. For each odd integer $j$
 between $1$ and $2n-1$ inclusive, let
 $\textrm{pairedWith}(j)$ be the even integer with which it is paired.
The DT code is 
the sequence $1, \textrm{pairedWith}(1), 3, 
\textrm{pairedWith}(3), \ldots, 2n-1, \textrm{pairedWith}(2n-1)$.

\section{The encoding}

The guiding intuition at work in the encoding is to think of a knot
diagram as representing the schematic of a signal flow. The crossings
represent gates or circuits. The arcs between the crossings represent
wires between the circuits. Thus, the whole encoding is really a
problem in how to represent this signal flow as a process.

This problem then decomposes into representing the crossing circuits
and wiring the circuits appropriately. As we will see in the sequel,
the crossing circuit of the $i$-th crossing of a knot, $K$, is a
process, $\meaningof{C(i)}_{\pi}(x_0,x_1,y_0,y_1,u)$, parameterized in $4+1$
ports corresponding to the in-coming and out-going arcs between each
crossing together with an additional port representing the
synchronizer for letting signals `cross over' in the gates. From this
perspective, the shape of the encoding may be expressed as follows.

\begin{equation}
  \meaningof{K}_{\pi} = (v_0 ... v_{2n-1}) \Pi_{i = 0}^{n-1} (\nu \; u)\meaningof{C(i)}_{\pi}(v_{\omega(i,0)},...,v_{\omega(i,3)},u)
\end{equation}

where $n$ is the crossing number of $K$, and $\omega: n \times 4 \to
2n$, gives the index into the list of ports, $v_0 ... v_{2n-1}$ used
to wire the crossing circuits together in a manner consistent with the
chosen knot diagram of $K$.

\subsection{An example: the trefoil as a process}
Before giving a formal account of the encoding we begin with an
example, showing how to represent the trefoil knot as a process.

% \scalebox{0.60}[0.60]{\includegraphics*[viewport=0 0 800 650]{BasicCrossingCircuit}}
% \begin{figure}
% \centering
% \scalebox{0.75}[0.75]{\includegraphics*[viewport=10 0 810 650]{BasicCrossingCircuit}}
% \caption{ Crossing circuit }
% \end{figure}

%\begin{figure}[tbp]
%\centering
%\scalebox{0.75}[0.75]{\includegraphics*[viewport=10 0 810 750]{TrefoilMethodIllustration}}
%\caption{ Trefoil as process }
%\end{figure}


\begin{figure}
\centering
\includegraphics[height=2in]{TrefoilMethodIllustration.pdf}
\caption{The figure illustrates encoding the trefoil as a
process}
\label{trefoil}
\end{figure}
The natural starting point is the process encoding of a crossing
circuit. To arrive at this construction we begin by observing that a
wire, $W(x_0,x_1)$, between two end-points, $x_0$ and $x_1$, is coded
as the process

\begin{equation}
  W(x_0,x_1) := x_0?(s).x_1!(s).W(x_0,x_1) + x_1?(s).x_0!(s).W(x_0,x_1)
\end{equation}

that receives a signal at either one of its end-points and sends it
out the other and then resumes being a wire.

Next, we note that a crossing circuit will have somewhat more
structure than a pair of wires between $x_0$ and $y_1$ and $x_1$ and
$y_0$. In particular, it must somehow code what it means for a wire to
cross over or under. We interpret this as a synchronization. The wire
crossing over is allowed to transmit the signal without waiting while
the under-crossing wire must wait for an additional input on a
synchronization channel. To alert the under-crossing wire that it may
now proceed, the over-crossing wire must fire off an output. Thus, a
crossing circuit is coded as the following process.

\begin{eqnarray}
  C(x_0,x_1,y_0,y_1,u) & := & x_1?(s).y_0!(s).(C(x_0,x_1,y_0,y_1,u)|u!) \nonumber \\
  & & + y_0?(s).x_1!(s).(C(x_0,x_1,y_0,y_1,u)|u!) \nonumber \\
  & & + x_0?(s).u?.y_1!(s).(C(x_0,x_1,y_0,y_1,u)) \nonumber \\
  & & + y_1?(s).u?.x_0!(s).(C(x_0,x_1,y_0,y_1,u)) 
\end{eqnarray}

Then the process encoding of the trefoil knot is essentially a
parallel composition of three crossing circuits. The main property we
must ensure is that the crossing circuits are connected to each other
in a way that respects the knot diagram. Additionally, we make each
synchronization channel local to each crossing via a restriction on
that channel.

\begin{eqnarray}
  \meaningof{K_3}_{\pi} & = & (v_0 ... v_{5}) (\nu \; u_0)C(v_0,v_1,v_2,v_3,u_0) \nonumber \\
  & & | (\nu \; u_1)C(v_2,v_3,v_2,v_4,u_5) \nonumber \\
  & & | (\nu \; u_2)C(v_4,v_5,v_0,v_1,u_1)
\end{eqnarray}

\subsection{The general encoding}

As we have already seen in the example how to code up a crossing
circuit, the only thing that remains is to show how to wire the
crossing circuits together in a manner consistent with a knot
diagram. To do this we make use of the DT code. We recognize that the
(lexicographically least) DT code is only unique for prime knots. This
is not really a limitation of our approach as we will demonstrate in a
subsequent section.


 {From a DT code, $DT(K)$, of a knot $K$ we recover a parity
     reversing involution $p: 2n \to 2n$, where $n$ is the crossing
     number of $K$, as well as an ordering on the crossings.}

{Observe that the process encoding of the knot will
consume
     exactly $2n$ ports for wiring up the $x0,...,y1$ positions of the
     crossing circuits.}

{Let $C(i)$ denote the $i$-th crossing of $K$ in the
ordering
     implied by the DT code, $DT(K)$. The constraints in the figure
     below (DT-constraints) uniquely determine a wiring of the
     crossing circuits, expressed in the map $\omega: n \times 4 \to
     2n$, that respects the connections of the crossings.
     \begin{figure}[tbp]
       \begin{eqnarray}
         y0(C(i)) & = & x0(C(i-1)) \\
         y1(C(i)) & = & x1(C(p(i)-1)) \\
         x0(C(i)) & = & y0(C(i+1)) \\
         x1(C(i)) & = & y1(C(p(i)+1))
       \end{eqnarray}
       \caption{DT-constraints}
   \end{figure}
    where we use $s_i(C(i))$ to denote the port used in the $s_i$
    position of the encoding of the $i$th crossing and all arithmetic is
    mod $2n$. }


 This result may be derived from a small tweak to Scharein's
 derivation of the graph sometimes called the knot shadow from a DT
 code.  

\begin{remark}
  We observe that it is possible to compress considerably the
  expression of these constraints. Let $s$ range over $\{x,y\}$, and
  define the involution $\hat{(-)}: \{x,y\} \to \{x,y\}$ by $\hat{x} =
  y$, $\hat{y} = x$, and let $\chi : \{x,y\} \to \{0,1\}$ be defined by
  $\chi(x) = 1$, $\chi(y) = 0$. Our constraints may be expressed by the
  single equation

  \begin{equation}
    s_i(C(i)) = \hat{s_i}(C(p^i(i) + (-1)^{\chi(s)+1}))
  \end{equation}
  
  The advantage of expressing the constraints in this more compressed
  way is that the two `switching conditions' are more prominently
  evident. Firstly, $x$ positions are always wired to $y$
  positions. Secondly, left-hand-side ports are always wired to
  predecessor (respectively sucessor) crossings using the index, while
  right-hand-side ports are always wired to predecessor (respectively
  successor) crossings using $p$ of the index.

  More generally, the encoding respects the discipline that both
  inter- and intra-crossing wires are always $x$ to $y$ while
  inter-crossing wires take $\__i$ positions to $\__i$ positions and
  intra-crossing wires take $\__i$ positions to $\__{i+1}$
  positions. In some very real sense, this discipline is the essence
  of what it means to cross.
\end{remark}

\section{Ambient isotopy as weak bisimilarity} \label{MainThm}

\begin{theorem}[main]
  Two knots, $K_0$ and $K_1$ are ambient isotopic, written here $K_0
  \sim K_1$, iff their encodings as processes are weakly bisimilar,
  i.e.
  \begin{equation}
    K_0 \sim K_1 \iff \meaningof{K_0}_{\pi} \simeq \meaningof{K_1}_{\pi}
  \end{equation}
\end{theorem}

\subsection{One direction}

In this section we show that if two knots are ambient isotopic then
their encodings as processes are bisimilar. Since two knot(
presentation)s are ambient isotopic if and only if there is a sequence
of Reidemeister moves transforming one presentation to the other
(\cite{sossinskyknotsandbraids}), it suffices to show that the
encodings of the Reidemeister moves as operations on processes
preserve bisimilarity.

\begin{lemma}[Reidemeister preserves bisimilarity] For each Reidemeister move, $\Omega_i$ if $K
  \stackrel{\Omega_i}{\longrightarrow} K'$ then $\meaningof{K}_{\pi} \simeq \meaningof{K'}_{\pi}$.
\end{lemma}

\subsubsection{Encoding the Reidemeister moves}

We need to add a few additional processes to our toolbelt. The next
one is a short-circuit. Its primary difference in behavior from a wire
is that it is also willing to offer a communication on the
synchronizer channel associated with a crossing circuit. As we will
see this behavior is useful for composing with a crossing circuit to
turn it into a wire.

\begin{eqnarray}
  S(x_0,x_1,u) & := & x_0?(s).x_1!(s).S(x_0,x_1,u) + x_1?(s).x_0!(s).S(x_0,x_1,u) \nonumber \\
  & & + u!.S(x_0,x_1,u)
\end{eqnarray}

\paragraph{$\Omega_1$}

We give $\meaningof{L(\Omega_1)}_{\pi}$
(resp. $\meaningof{R(\Omega_1)}_{\pi}$), the process encodings of the
left (resp. right) hand side of the Reidemeister move, $\Omega_1$.

\begin{eqnarray}
  \meaningof{L(\Omega_1)}_{\pi}(x_0,x_1) & := & (\nu \; z_0 z_1 u_0)(S(z_0,z_1,u_0) | C(z_0,z_1,x_0,x_1,u_0)) \\
  \meaningof{R(\Omega_1)}_{\pi}(x_0,x_1) & := & W(x_0,x_1)
\end{eqnarray}

It is straightforward to calculate the bisimulation that verifies

\begin{equation}
  \meaningof{L(\Omega_1)}_{\pi}(x_0,x_1) \simeq \meaningof{R(\Omega_1)}_{\pi}(x_0,x_1)
\end{equation}

\paragraph{$\Omega_2$}

\begin{eqnarray}
  \meaningof{L(\Omega_2)}_{\pi}(x_0,x_1,y_0,y_1) & := & (\nu \; z_0 z_1 u_0)(C(x_0,x_1,z_0,z_1,u_0) \nonumber \\
  & & | C(z_0,z_1,y_0,y_1,u_0)) \\
  \meaningof{R(\Omega_2)}_{\pi}(x_0,x_1,y_0,y_1) & := & W(x_0,y_0) | W(x_1,y_1)
\end{eqnarray}

Likewise that

\begin{equation}
  \meaningof{L(\Omega_2)}_{\pi}(x_0,x_1,y_0,y_1) \simeq \meaningof{R(\Omega_2)}_{\pi}(x_0,x_1,y_0,y_1)
\end{equation}

is a straightforward calculation.

\paragraph{$\Omega_3$}

\begin{eqnarray}
  \meaningof{L(\Omega_3)}_{\pi}(x_0,x_1,y_0,y_1,w_0,z_0) & := & (\nu \; u_0 v_0 w_1 z_1)(C(x_0,x_1,v_0,z_1,u_0) \nonumber \\
  & & | C(w_0,v_0,z_0,w_1,u_0) \nonumber \\
  & & | C(w_1,z_1,y_0,y_1,u_0)) \\
  \meaningof{R(\Omega_3)}_{\pi}(x_0,x_1,y_0,y_1,w_0,z_0) & := & (\nu \; u_0 v_0 w_1 z_1)(C(x_0,x_1,v_0,z_1,u_0) \nonumber \\
  & & | C(v_0,w_1,y_0,y_1,u_0) \nonumber \\
  & & | C(z_1,w_0,w_1,z_0,u_0)) \\
\end{eqnarray}

Again, straightforward, if tedious calculation will show that

\begin{equation}
  \meaningof{L(\Omega_2)}_{\pi}(x_0,x_1,y_0,y_1,w_0,z_0) \simeq \meaningof{R(\Omega_2)}_{\pi}(x_0,x_1,y_0,y_1,w_0,z_0)
\end{equation}

\begin{remark}
  Up to this point, when referring to the encoding of a knot as a
  process, we have been ignoring the fact that a given knot may have
  many diagrams. Because the Reidemeister moves (as operations on
  processes) preserve bisimilarity this ignorance can remain a blissful
  state.
\end{remark}

\subsection{The other direction}

The intuitive argument runs as follows. Suppose $K_0, K_1$ such that
$\meaningof{K_0}_{\pi} \sim \meaningof{K_1}_{\pi}$. For any crossing
$C_i(x_{0},x_{1},y_{0},y_{1},u)$ of $\meaningof{K_{i \bmod 2}}_{\pi}$
there are exactly four transitions possible. W.l.o.g. we may consider
only two of them: one for the over-crossing and one for the
under-crossing. Of the two, only one of them (the over-crossing
transition) immediately enables a transition in another
crossing. Because of the DT-constraints we know that only one such
crossing is so enabled, and only one of its transition is enabled. We
think of this as the `successor' crossing of the current one. Note
that the DT-constraints do not determine whether the over- or
under-crossing is enabled, but \emph{eventually} (i.e. in the presence
of at least two signals or pulses to the knot-process) the successor
crossing enables its successor. The DT-constraints ensure that we may
continue in this way visiting every crossing exactly twice.

Because $\meaningof{K_{i}}_{\pi} \sim \meaningof{K_{i+1 \bmod
    2}}_{\pi}$, $\meaningof{K_{i+1 \bmod 2}}_{\pi}$ must mimick the
path we have traced. Further, bisimulation is symmetric and so our
argument may be run from $\meaningof{K_{i+1 \bmod 2}}_{\pi}$ to
$\meaningof{K_{i}}_{\pi}$. Thus, the bisimulation establishes a
bijection between the crossings of $K_{0}$ and $K_{1}$ that preserves
the polarity (i.e. over/under relationships) and the connections
between the crossings. Thus, $K_{0}$ is in the same ambient isotopy
class as $K_{1}$.

\section{Discussion}

\subsection{From prime knots to composites via parallel composition}

We can compose knots in the process representation by rewiring
crossings in a manner that respects the DT-constraints. 

Thus if we have

\begin{eqnarray}
  \meaningof{K_0}_{\pi} & = & (v_0 ... v_{2n_{0}-1}) \Pi_{i = 0}^{n_{0}-1} (\nu \; u)\meaningof{C_0(i)}_{\pi}(v_{\omega(i,0)},...,v_{\omega(i,3)},u) \\
  \meaningof{K_1}_{\pi} & = & (x_0 ... x_{2n_{1}-1}) \Pi_{i = 0}^{n_{1}-1} (\nu \; u)\meaningof{C_1(i)}_{\pi}(x_{\omega(i,0)},...,x_{\omega(i,3)},u)
\end{eqnarray}

and we want to wire
$\meaningof{C_0(k)}_{\pi}(v_{\omega(k,0)},v_{\omega(k,1)},v_{\omega(k,2)},v_{\omega(k,3)},u_i)$
to


$\meaningof{C_1(k')}_{\pi}(x_{\omega(k',0)},x_{\omega(k',1)},x_{\omega(k',2)},x_{\omega(k',3)},u_j)$
we can construct a new knot

\begin{eqnarray} \label{knotcomp}
  \meaningof{K_0 + K_1}_{\pi} & = & ( w_0 w_1)( v_0 ... \widehat{v_{\omega(i,2)}} ... \widehat{v_{\omega(i,3)}} ..., v_{2n_{0}-1})( x_0 ... \widehat{x_{\omega(i,0)}} ... \widehat{x_{\omega(i,1)}} ... x_{2n_{1}-1}) \nonumber \\
  & & \Pi_{i = 0}^{k-1} (\nu \; u)\meaningof{C_0(i)}_{\pi}(v_{\omega(i,0)},...,v_{\omega(i,3)},u) \nonumber \\
  & & | \Pi_{i = k+1}^{n_{0}-1} (\nu \; u)\meaningof{C_0(i)}_{\pi}(v_{\omega(i,0)},...,v_{\omega(i,3)},u) \nonumber \\
  & & | \meaningof{C_0(k)}_{\pi}(v_{\omega(k,0)},v_{\omega(k,1)},w_0,w_1,u_i) | \meaningof{C_1(k')}_{\pi}(w_0,w_1,x_{\omega(k',2)},x_{\omega(k',3)},u_j) \nonumber \\
  & & | \Pi_{i = 0}^{k'-1} (\nu \; u)\meaningof{C_1(i)}_{\pi}(x_{\omega(i,0)},...,x_{\omega(i,3)},u) \nonumber \\
  & & | \Pi_{i = k'+1}^{n_{1}-1} (\nu \; u)\meaningof{C_1(i)}_{\pi}(x_{\omega(i,0)},...,x_{\omega(i,3)},u)
\end{eqnarray}

where $\widehat{\alpha_{\omega(a,b)}}$ denotes the deletion of that name from the sequence.

Taking advantage of the compositional nature of the encoding, and
recalling equation \ref{linkingident} in \ref{linkingabs} we see that

% For $\meaningof{K}_{\pi} = ( v_0 ... v_{2n-1})
% \Pi_{i = 0}^{n-1} (\nu \;
% u)\meaningof{C(i)}_{\pi}(v_{\omega(i,0)},...,v_{\omega(i,3)},u)$ we write

% \begin{eqnarray}
%   & K - j \triangleq ( v_0 ... v_{2n-1}) & \nonumber \\
%   & \Pi_{i = 0}^{j-1} (\nu \; u)\meaningof{C(i)}_{\pi}(v_{\omega(i,0)},...,v_{\omega(i,3)},u) & \nonumber \\
%   & | \Pi_{i = j+1}^{n-1} (\nu \; u)\meaningof{C(i)}_{\pi}(v_{\omega(i,0)},...,v_{\omega(i,3)},u) & 
% \end{eqnarray}

% Thus,

% \begin{eqnarray}
%   \meaningof{K}_{\pi} & \equiv & ( w_0 ... w_{2n-1})((K - j)\langle w_0 ... w_{2n-1} \rangle \nonumber \\
%   & & | (\nu \; u)\meaningof{C(j)}_{\pi}(w_{\omega(j,0)},...,w_{\omega(j,3)},u))
% \end{eqnarray}

% Adapting a convention of Milner \cite{milner91polyadicpi} we shorten
% this further, arriving at

\begin{eqnarray}
  \meaningof{K}_{\pi} & \equiv & (\meaningof{K}_{\pi} - \meaningof{C(j)}_{\pi}) \# \meaningof{C(j)}_{\pi}
\end{eqnarray}

(eliding the restriction scope on $\meaningof{C(j)}_{\pi}$ and its arguments).

Similarly, we can provide a more compact notation for wiring two crossing circuits together.

\begin{eqnarray}
  & \meaningof{C_0(k)}_{\pi} \smallsmile \meaningof{C_1(k')}_{\pi} & \nonumber \\
  & \triangleq & \nonumber \\
  & ( w_0 w_1)\meaningof{C_0(k)}_{\pi}(v_{\omega(k,0)},v_{\omega(k,1)},w_0,w_1,u_i) & \nonumber \\
  & | \meaningof{C_1(k')}_{\pi}(w_0,w_1,x_{\omega(k',2)},x_{\omega(k',3)},u_j) &
\end{eqnarray}

allowing us to write \ref{knotcomp} as

\begin{eqnarray}
  & \meaningof{K_0 + K_1}_{\pi} & \nonumber \\
  & = & \nonumber \\
  & (\meaningof{K_0}_{\pi} - \meaningof{C_0(k)}_{\pi}) \# (\meaningof{C_0(k)}_{\pi} \smallsmile \meaningof{C_1(k')}_{\pi}) \# (\meaningof{K_1}_{\pi} - \meaningof{C_1(k')}_{\pi}) &
\end{eqnarray}

This construction not only illustrates, as previously promised, that
the method of encoding is not restricted to prime knots, but that
there is a good conceptual fit between the domain and the
representation: the notion of composition of knots lines up with a
compositor, namely parallel composition, of processes. It should be
mentioned in this connection that there is nothing in the proof of the
main theorem that in any way depends on primality of the knots being
compared.

\subsection{Old invariants from new: the Kauffman bracket}

We also observe that the representation is particularly natural for
working with skein relations as in the Kauffman bracket. In
particular, we may recursively define a function, $\langle\!\langle - \rangle\!\rangle$, taking a
process, $P$ in the image of our encoding of knots to a Laurent
polynomial (the Kauffman bracket of the knot $P$ encodes).

To make the encoding more intuitive and well-aligned with the standard
definition it will be useful to add a few more items to our toolbox. First, we have the following
operators on crossing circuits

\begin{eqnarray}
  C^{\asymp}(j) & \triangleq & W(v_{\omega(j,0)},v_{\omega(j,1)}) | W(v_{\omega(j,2)},v_{\omega(j,3)}) \nonumber \\
  C^{)(}(j) & \triangleq & W(v_{\omega(j,0)},v_{\omega(j,2)}) | W(v_{\omega(j,1)},v_{\omega(j,3)})
\end{eqnarray}

representing the two ways to wire the ports of the crossing circuit to avoid the crossing.

Next, we characterize a \emph{cycle}, which will represent for us the
class of processes corresponding to the unknot, as any process $R$,
structurally equivalent to a process of the form $\Pi_{i = 0}^{k-1}
W(v_i,v_{i+1 \bmod k})$, for some $k$.

% \begin{eqnarray}
%   & \langle\!\langle( v_0 ... v_{2n-1}) (\nu \; u)\meaningof{C(k)}_{\pi}(v_{\omega(k,0)},...,v_{\omega(k,3)},u) & \nonumber \\
%   & | \Pi_{i = 0}^{k-1} (\nu \; u)\meaningof{C(i)}_{\pi}(v_{\omega(i,0)},...,v_{\omega(i,3)},u) & \nonumber \\
%   & | \Pi_{i = k+1}^{n-1} (\nu \; u)\meaningof{C(i)}_{\pi}(v_{\omega(i,0)},...,v_{\omega(i,3)},u)\rangle\!\rangle & \nonumber \\
%   & \triangleq & \nonumber \\
%   & A \cdot \langle\!\langle( v_0 ... v_{2n-1}) (\nu \; u) W(v_{\omega(k,0)},v_{\omega(k,1)}) | W(v_{\omega(k,2)},v_{\omega(k,3)}) & \nonumber \\
%   & | \Pi_{i = 0}^{k-1} (\nu \; u)\meaningof{C(i)}_{\pi}(v_{\omega(i,0)},...,v_{\omega(i,3)},u) & \nonumber \\
%   & | \Pi_{i = k+1}^{n-1} (\nu \; u)\meaningof{C(i)}_{\pi}(v_{\omega(i,0)},...,v_{\omega(i,3)},u)\rangle\!\rangle & \nonumber \\
%   & + A^{-1} \cdot \langle\!\langle( v_0 ... v_{2n-1}) (\nu \; u) W(v_{\omega(k,0)},v_{\omega(k,2)}) | W(v_{\omega(k,1)},v_{\omega(k,3)}) & \nonumber \\
%   & | \Pi_{i = 0}^{k-1} (\nu \; u)\meaningof{C(i)}_{\pi}(v_{\omega(i,0)},...,v_{\omega(i,3)},u) & \nonumber \\
%   & | \Pi_{i = k+1}^{n-1} (\nu \; u)\meaningof{C(i)}_{\pi}(v_{\omega(i,0)},...,v_{\omega(i,3)},u)\rangle\!\rangle &
% \end{eqnarray}

Then our function, $\langle\!\langle - \rangle\!\rangle$, is determined by the constraints

\begin{enumerate}
  \item \begin{eqnarray}
      \langle\!\langle(K-j) \# C(j)\rangle\!\rangle & = & a \langle\!\langle(K-j) \# C^{\asymp}(j)\rangle\!\rangle  + a^{-1} \langle\!\langle(K-j) \# C^{)(}(j)\rangle\!\rangle
    \end{eqnarray}

  \item for any cycle, $R$, that 
    \begin{equation}
      \langle\!\langle( v_0 ... v_{2n-1})R\rangle\!\rangle = 1
    \end{equation}

  \item \begin{eqnarray}
      \langle\!\langle( v_0 ... v_{2n-1})(P | R)\rangle\!\rangle & = & (-a^2 - a^{-2})\langle\!\langle( v_0 ... v_{2n-1})P\rangle\!\rangle
    \end{eqnarray}
  \end{enumerate}

  corresponding to the usual equations defining the Kauffman bracket:

\begin{enumerate}
\item $\bigl<
   \xygraph{              % The crossing
     !{0;/r1.0pc/:}
     [u(0.5)]
     !{\xunderv}
   }
 \bigr> 
 =
 a\bigl<\hspace{0.1pc} % The horizontal crossing eliminated
   \xygraph{
     !{0;/r1.0pc/:}
     [u(0.5)]
     !{\xunoverv}
   }\hspace{0.1pc}
 \bigr> 
 +
 a^{-1}\bigl<\hspace{0.1pc} % the vertical crossing eliminated
 \xygraph{
  !{0;/r1.0pc/:}
       [u(0.5)]
  !{\xunoverh}
}\hspace{0.1pc}
\bigr>
 $

 \item $ \bigl<\hspace{0.1pc} 
 \xygraph{                          % the unknot
 !{0;/r1.0pc/:}
 !{\vcap-}
 !{\vcap+}
 }
 \hspace{0.1pc}
 \bigr>
 =1
 $
 
 \item $
 \bigl<\hspace{0.1pc}
 L \bigsqcup 
  \xygraph{               % the unknot, again
 !{0;/r1.0pc/:}
 !{\vcap-}
 !{\vcap+}
 }
 \hspace{0.1pc}\bigr>
 =
 (-a^{2}-a^{-2})\bigr<\hspace{0.1pc}L\hspace{0.1pc}\bigr>$
\end{enumerate}

% \begin{enumerate}
% \item $q^{-1}J(L^+) - qJ(L^-) = (q^{\frac{1}{2}}-q^{\frac{-1}{2}})J(L^o) $ (the skein relation)

% \item $J(L\cup \bigcirc)= -(q^{-{1}\over{2}}+q^{{1}\over{2}})J(L)$

% \item $J(\bigcirc) = 1$
% \end{enumerate}

\section{Conclusions and future work}

\subsection{Generalizing crossings and virtual knots}

Kauffman posits an intriguing new member to the knot family by
virtualizing the crossings in a knot diagram. We note that while we
arrived at this encoding before becoming aware of Kauffman's work,
this line of investigation is very much in line with the intuitions
guiding the encoding presented here. Specifically, we see no reason
why the crossing circuit cannot be any $\pi$-calculus process that
respects the interface of crossing circuit presented here.

\subsection{Braids and tangles}

As may be seen from the encoding of the Reidemeister moves, nothing in
this approach restricts it to knots. In particular, the same
techniques may be lifted to braids and tangles.

\subsection{Knot invariants as program invariants}

To what class of programs do knots belong? Can various knot invariants
be usefully employed as quick calculations of program equivalence for
that class of programs?

\subsection{Other calculi, other bisimulations and geometry as behavior}

Of course, the astute reader may have noticed that the encoding
described here is not much more than a linear notation for a minimal
graph-based representation of a knot diagram. In this sense, there is
nothing particularly remarkable about the representation. Though
expressed in a seemingly idiosyncratic way, it's more or less the same
information that any freely downloadable program for calculating knot
polynomials uses routinely. What is remarkable about this
representational framework is that it enjoys an \emph{independent}
interpretation as the description of the behavior of concurrently
executing processes. Moreover, the notion of the equivalence of the
behavior of two processes (in the image of the encoding) coincides
exactly with the notion of ambient isotopy. It is the precise
alignment of independently discovered notions that often indicates a
phenomena worth investigating.

This line of thought seems particularly strengthened when we recall,
as we did in the introduction, the $\pi$-calculus is just one of many
`computational calculi' that may be thought of as an
\emph{algebra+computational dynamics} and that virtually every such calculus
is susceptible to a wide range of bisimulation and bisimulation up-to
techniques. As such, we see the invariant discussed here as one of
many potential such invariants drawn from these relatively new
algebraic structures. It is in this sense that we see it as a new kind
of invariant and is the inspiration for the other half of the title of
this paper.

Finally, if the reader will permit a brief moment of philosophical
reflection, we will conclude by observing that such a connection fits
into a wider historical context. There is a long-standing enquiry and
debate into the nature of physical space. Using the now familiar
signposts, Newton's physics -- which sees space as an absolute
framework -- and Einstein's -- which sees it as arising from and
shaping interaction -- we see this connection as fitting squarely
within the Einsteinian weltenschauung. On the other hand, we cannot
help but notice that unlike the particular mathematical framework in
which Einstein worked out his programme -- a framework that required
continuity -- behavior and the implied notions of space and time are
entirely discrete in this setting, built out of names and acts of
communication. In this light we look forward to revisiting the now
well-established connection between the various knot invariants such
as the Jones polynomial and quantum groups.


\paragraph{Acknowledgments.}
The author wishes to acknowledge his longstanding debt to Samson
Abramksy for making so accessible his foundational insights into the
Curry-Howard isomorphism.

% ------------------------------------------------------------------------
%GATHER{Xbib.bib}   % For Gather Purpose Only
%GATHER{Paper.bbl}  % For Gather Purpose Only
\bibliographystyle{plain}
\bibliography{main}

% ------------------------------------------------------------------------

% ------------------------------------------------------------------------

\end{document}
% ------------------------------------------------------------------------
