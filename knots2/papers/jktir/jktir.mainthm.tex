\subsection{The distinguishing power of dynamics}\label{sub:dynamic distinction} % (fold)

In summary, to each knot $\mathcal{K}$ the encoding associates an
invariant $\meaningof{K}$, an expression in a calculus of
message-passing processes via an encoding of a diagram of the
knot. More precisely, given the set of knot diagrams $\mathbb{K}$ and
the set of processes modulo structural equivalence $\mathbb {P}$ (see section
\ref{sub:the_syntax_and_semantics_of_the_notation_system}), the encoding induces a map, $\meaningof{-}: \mathbb{K} \to \mathbb
{P}$. Most importantly, the notion of equivalence of knots coincides perfectly with the notion of equivalence of
processes, i.e. bisimulation (written here and in the sequel
$\simeq$). Stated more formally,

\begin{theorem}[main]
\begin{eqnarray*}
    K_1 \sim K_2 & \iff & \meaningof{K_1} \simeq \meaningof{K_2}. \nonumber
\end{eqnarray*}
\end{theorem}

%Thus $\mathbb{K}$ may be taken to be the collection of isotopy classes of knots. 
In particular, in contrast to other invariants, the alignment of process
dynamics with knot characteristics is what enables the invariant
identified here to be perfectly distinguishing. As discussed below,
among the other beneficial consequences of this alignment is the
ability to apply process logics, especially the spatial sub-family of
the Hennessy-Milner logics, to reason about knot characteristics and
knot classes.

We see moreover the possibility of a deeper connection. As mentioned
in the previous section, bisimulation has turned out to a powerful
proof technique in the theory of computation adaptable to a wide range
of situations and admitting a number of potent up-to techniques
\cite{DBLP:conf/lics/Sangiorgi04}. One of the central aims of this
research is to broaden the domain of applicability of
bisimulation-based proof methods.

% subsection basic_interpretation (end)