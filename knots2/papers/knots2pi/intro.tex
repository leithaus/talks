\section{Introduction}\label{sec:introduction} 

Recent research in concurrency theory has led to the consideration of
geometric interpretations of concurrent computation such as Goubault's
investigations of higher-order automata
\cite{DBLP:journals/mscs/Goubault00a,DBLP:conf/concur/GoubaultJ92} and Herlihy's application of
homology theory \cite{DBLP:journals/entcs/HerlihyRT02}. Here, the authors exploit a duality hidden within that framework,
switching the roles of domains
and change the
tool of investigation, applying it to a topological investigation. 
More specifically, in this paper we use the \pic\
 to construct an isotopy invariant of knots,
giving a method to directly describe any
given knot $K$
as a process $\meaningof{K}_{\pi}$ within the\ \pic. The encoding description is dependent upon the
knot presentation. Nonetheless the encoding has
a  nice property: 
\begin{theorem}[Main Theorem]
  Two knots, $K_0$ and $K_1$ are ambient isotopic, written here $K_0
  \sim K_1$, iff their encodings as processes are weakly bisimilar,
  i.e.
  
  \[ % Don't number this equation
    K_0 \sim K_1 \iff \meaningof{K_0}_{\pi}
\simeq \meaningof{K_1}_{\pi}
  \]
\end{theorem}

While this makes clear that the
encoding is a strong invariant of knots up to ambient isotopy,
 we see a broader motivation for investigating knots from
this perspective. Bisimulation has proven to be a
powerful and flexible proof principle, adaptable to a wide
range of situations and admitting a few
significant,
potent equivalence ("up-to") techniques
\cite{DBLP:conf/mfcs/Sangiorgi95,DBLP:conf/lics/Sangiorgi04}. 
Thus, we
seek to apply  to
topological inquiries what we have learned
from the past
several decades of investigation into notions of
the equivalence of computational behavior.
The dual notion, laid bare here in its
simplest-to-grasp context,
may bear
fruit in the study of the equivalence of spaces
which have simplicial models.

In section \ref{Pinut} we give a brief review of the polyadic $\pi$-calculus,
immediately following that with a review of the primary results from
knot theory needed to state and prove our main theorem. Next, we
introduce the intuitions behind the encoding, sketching the general
shape and walking through the procedure in the case of the trefoil. We
follow this with a section on the details of the
encoding. This puts us in a
position to state and prove the main theorem in section
\ref{MainThm}. In the penultimate section we discussion some of the results
following from this method of encoding, illustrating a way to
interpret knot composition as parallel composition and a way to
interpret the Kauffman bracket (and hence a number of other knot
invariants) in this setting. In the final section
we state some conclusions and
foreshadow some directions for future research.



